% Twoside implica che i capitoli inizino sempre con la prima pagina a sinistra, eventualmente lasciando una pagina vuota nel capitolo precedente. 
\documentclass[a4paper, twoside,openright]{report}

%spazio interlinea
\usepackage{setspace}
\onehalfspacing
% Dimensione dei margini
\usepackage[a4paper,top=3cm,bottom=3cm,left=3cm,right=3cm]{geometry} 
% Dimensione del font
\usepackage[fontsize=13pt]{scrextend}
% Lingua del testo
\usepackage[english,italian]{babel}
% Lingua per la bibliografia
\usepackage[fixlanguage]{babelbib}
% Codifica del testo
\usepackage[utf8]{inputenc} 
% Encoding del testo
\usepackage[T1]{fontenc}
%font
\usepackage{helvet}

% Kerinig del testo
\usepackage[expansion=false]{microtype}
\SetTracking{encoding = *, shape = sc}{30}
% Per modificare l'header delle pagine 
\usepackage{fancyhdr}               

\usepackage{float}

% Librerie matematiche
\usepackage{amssymb}
\usepackage{amsmath}
\usepackage{amsthm}         

% Uso delle immagini
\usepackage{graphicx}
% Uso dei colori
\usepackage[dvipsnames]{xcolor}         
% Uso dei listing per il codice
\usepackage{listings} 
\lstset{
  backgroundcolor=\color{gray!10},   % colore dello sfondo
  basicstyle=\ttfamily,              % imposta il font del testo
  breaklines=true,                   % interrompi le linee troppo lunghe
  frame=single,                      % aggiungi una cornice intorno al codice
  captionpos=b,                      % posiziona la didascalia in basso
  numbers=left,                      % mostra i numeri di linea a sinistra
  numberstyle=\small,                % imposta lo stile del numero di linea
  showstringspaces=false,            % non mostrare gli spazi nelle stringhe come underscore
  escapeinside={(*@}{@*)},           % se vuoi inserire LaTeX all'interno del tuo codice
}
         
% Per inserire gli hyperlinks tra i vari elementi del testo 
\usepackage{hyperref}     
% Diversi tipi di sottolineature
\usepackage[normalem]{ulem}

% -----------------------------------------------------------------

% Modifica lo stile dell'header
\pagestyle{fancy}
\fancyhf{}
\lhead{\rightmark}
\rhead{\textbf{\thepage}}
\fancyfoot{}
\setlength{\headheight}{16pt}

% Rimuove il numero di pagina all'inizio dei capitoli
\fancypagestyle{plain}{
  \fancyfoot{}
  \fancyhead{}
  \renewcommand{\headrulewidth}{0pt}
}

\definecolor{backcolour}{rgb}{0.90,0.95,0.92}

% Stile del codice
% \lstset{style=codeStyle}
\lstdefinestyle{codeStyle}{
    backgroundcolor=\color{backcolour},
    commentstyle=\color{teal},
    keywordstyle=\color{Magenta},
    numberstyle=\tiny\color{gray},
    stringstyle=\color{violet},
    basicstyle=\ttfamily\scriptsize,
    breakatwhitespace=false,     
    breaklines=true,                 
    captionpos=b,                    
    keepspaces=true,                 
    numbers=left,                    
    numbersep=5pt,                  
    showspaces=false,                
    showstringspaces=false,
    showtabs=false,
    tabsize=1
} \lstset{style=codeStyle}

% \lstset{style=longBlock}
\lstdefinestyle{longBlock}{
    commentstyle=\color{teal},
    keywordstyle=\color{Magenta},
    numberstyle=\tiny\color{gray},
    stringstyle=\color{violet},
    basicstyle=\ttfamily\scriptsize,
    breakatwhitespace=false,         
    breaklines=true,                 
    captionpos=b,                    
    keepspaces=true,                 
    numbers=left,                    
    numbersep=5pt,                  
    showspaces=false,                
    showstringspaces=false,
    showtabs=false,                  
    tabsize=2
} \lstset{style=codeStyle}

% Togliendo il commento al comando che segue, si inseriscono nella bibliografia anche le fonti presenti in Bibliography.bib ma non citati direttamente con il comando \cite
\nocite{*}

% Margini prima e dopo blocchi di codice, per avere più distanza
\lstset{aboveskip=20pt,belowskip=20pt}

% Modifica dello stile dei riferimenti
\hypersetup{
    colorlinks,
    linkcolor=black,
    citecolor=black
}

% Aggiunti definizioni, teoremi, linea e listing
\newtheorem{definition}{Definizione}[section]
\newtheorem{theorem}{Teorema}[section]
\providecommand*\definitionautorefname{Definizione}
\providecommand*\theoremautorefname{Teorema}
\providecommand*{\listingautorefname}{Listing}
\providecommand*\lstnumberautorefname{Linea}

\raggedbottom



% -----------------------------------------------------------------
\begin{document}
\begin{titlepage}
\begin{figure}[!htb]
    \centering
\end{figure}
\vspace{30mm}
\begin{center}
    \LARGE{UNIVERSITÀ DEGLI STUDI DI MODENA E REGGIO EMILIA}
    \vspace{5mm}
    \\ \large{DIPARTIMENTO DI INGEGNERIA "ENZO FERRARI"}
    \vspace{5mm}
    \\ Laurea Triennale in Ingegneria Informatica
\end{center}

\vspace{15mm}
\begin{center}
    {\LARGE{\bf Planning locale\\ \vspace{5mm} di una vettura da Formula Student\\ \vspace{5mm} a guida autonoma}}
    
    % Se il titolo è abbastanza corto da stare su una riga, si può usare
    
    % {\LARGE{\bf Un fantastico titolo per la mia tesi!}}
\end{center}
\vspace{30mm}

\begin{minipage}[t]{0.47\textwidth}
	{\large{Relatore:}{\normalsize\vspace{3mm}
	\bf\\ \large{Prof: Francesco Guerra}}}
\end{minipage}
\hfill
\begin{minipage}[t]{0.47\textwidth}\raggedleft
	{\large{Candidato:}{\normalsize\vspace{3mm} \bf\\ \large{Filippo Gibertini}}}
\end{minipage}

\vspace{30mm}
\hrulefill
\\\centering{\large{Anno Accademico 2022/2023}}

\end{titlepage}
\let\cleardoublepage\clearpage
\include{chapters/Abstract}
\let\cleardoublepage\clearpage

\tableofcontents

\listoffigures

\chapter{introduzione}
Nell'ambito della guida autonoma il Planning è una componente software che si occupa di delineare la traiettoria che la macchina deve seguire. In particolare, nelle auto a guida autonoma da competizione, il Planning si suddivide in Locale e Globale. Il Planning Globale è specializzato nel navigare attraverso percorsi conosciuti, mentre il Planning Locale si occupa di generare la traiettoria della macchina attraverso percorsi di cui non si posseggono informazioni pregresse. Gli aspetti cruciali di cui si è dovuto tenere conto nella realizzazione dell'algoritmo di Planning Locale sono: \textit{(1)} la molteplicità delle prove di gara che si devono affrontare, ognuna delle quali necessita di un Planning Locale dedicato, \textit{(2)} la corretta comunicazione con gli altri elementi software (nodi) che compongono l'algoritmo di guida autonoma, \textit{(3)} la necessità di un algoritmo che si avvicini il più possibile ad essere real-time.

\include{chapters/state_of_the_art}
\include{chapters/base_knowledge}
\chapter{Implementazione}
In questo capitolo verrà descritto come è sono stati tradotti in codice gli algoritmi descritti nei capitoli precedenti.
E' stato implementato un planner per ogni prova della gara: \index{1}acceleration, \index{2}skidpad, \index{3}autocross e trackdrive, che condividendo lo stesso tracciato condividono lo stesso planner. 
Ogni planner viene inizializzato all'interno di un nodo ROS che ha anche le responsabilità di reperire il tipo della prova e istanziare  publishers e subscribers.
\section{Local planner node}
Il nodo è stato definito nel file \texttt{local\_planner\_node.h} ed è stato implementato nel file \texttt{local\_planner\_node.cpp}. Il nodo si prende a carico le seguenti responsabilità:
\begin{itemize}
	\item Reperire i parametri relativi ai topic e il parametro relativo al tipo di evento dal file di configurazione.
	\item Inizializzare i publishers che sono comuni a tutte le prove.
	\item Inizializzare il planner corrispondente al tipo di evento contenuto nel parametro "node/eventType". 
	\item inizializzare i subscribers specifici facendo il bind fra le callback del planner e i topic necessari.
\end{itemize} 
\include{chapters/project}
\chapter{Conclusioni}

L'applicazione, descritta all'interno di questa elaborato, ovviamente non è ultimata ma è solamente un punto di partenza per l'introduzione della programmazione modulare all'interno della libreria WLDT.   Difatti il progetto può essere, sicuramente,  migliorato ed ampliato sviluppando diverse tipologie di DigitalAdapter e PhysicalAdapter “complessi” (HTTPDigitalAdapter e MQTTPhysicalAdapter). Inoltre la flessibilità e la dinamicità tra le varie componenti del Digital Twin può essere, ulteriormente, migliorata riducendo la dipendenza tra il bundle rappresentante il modello del DT ed il bundle della ShadowingFunction, permettendo al Digital Twin di utilizzare più ShadowingFunction senza resettare lo stato interno. Infine potrebbe essere sviluppato anche un ulteriore bundle, che attraverso l'utilizzo di un event bus, gestisca la comunicazione tra i vari componenti del Digital Twin.
\bibliographystyle{unsrt}
\bibliography{main}
\end{document}