\chapter{introduzione}
Nell'ambito della guida autonoma il Planning è una componente software che si occupa di delineare la traiettoria che la macchina deve seguire. In particolare, nelle auto a guida autonoma da competizione, il Planning si suddivide in Locale e Globale. Il Planning Globale è specializzato nel navigare attraverso percorsi conosciuti, mentre il Planning Locale si occupa di generare la traiettoria della macchina attraverso percorsi di cui non si posseggono informazioni pregresse. Gli aspetti cruciali di cui si è dovuto tenere conto nella realizzazione dell'algoritmo di Planning Locale sono: \textit{(1)} la molteplicità delle prove di gara che si devono affrontare, ognuna delle quali necessita di un Planning Locale dedicato, \textit{(2)} la corretta comunicazione con gli altri elementi software (nodi) che compongono l'algoritmo di guida autonoma, \textit{(3)} la necessità di un algoritmo che si avvicini il più possibile ad essere real-time.
